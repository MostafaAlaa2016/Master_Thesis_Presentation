\section{Network Architecture and Training}

%%%%%%%%%%%%%%%%%%%%%%%%%%%%%%%%%%%%%%%%%%%%%%%%%%
\begin{frame}[fragile]{Which Network?}


		\begin{itemize}
			\item[--] \textbf{Pattern} is a sequence of characters.
			\item  Unlike feedforward neural networks, RNNs can use their internal state (memory) to process sequences of inputs.
			\item In theory, RNNs are capable of handling long-term dependencies. 	However, in practice they do not, due to the \alert{exploding gradient problem}
			\item LSTMs was designed to solve the long-term dependency problem using internal memory gates.
		\end{itemize}
		
\end{frame}
%%%%%%%%%%%%%%%%%%%%%%%%%%%%%%%%%%%%%%%%%%%%%%%%%%
%%% Nueral Networks
%%%%%%%%%%%%%%%%%%%%%%%%%%%%%%%%%%%%%%%%%%%%%%%%%%
\begin{frame}[fragile]{Neural Networks Architectures}
\Wider{	\begin{columns}
	
	\begin{column}{.45\textwidth}

\begin{center}
	\begin{figure}[t]
		\tikzset{%
	neuron missing/.style={
		draw=none, 
		scale=2,
		text height=0.1cm,
		execute at begin node=\color{black}$\vdots$
	},
}

\begin{tikzpicture}[x=1.2cm, y=2cm, >=stealth,scale=.7, transform shape]
  % The grid
% \draw[step=0.5, gray!40, very thin] (-1,-3) grid (4,2);

%\node (A) at (.4, 1.8){$L_1$} ;
%\node[right,xshift=0.2cm] (B) at (3.5, 1.8) {$L_n$} ;
% arrows
%\draw[->, to path={-| (\tikztotarget)}] (A) edge (B) ;

%node[left, xshift=0.0cm, yshift=0.0cm,fill=orange!25,text width=1.3cm] {{\scriptsize Cell Type (LSTM, GRU)}};

%\draw[arrows=-latex]  (0,1.5) -- (-1.1,1)  node[left, xshift=0.0cm, yshift=0.05cm,fill=orange!25,text width=1.3cm] {{\scriptsize Cell Units (nU)}};

%\draw[arrows=-latex]  (0,1.5) -- (-1.1,2.2)  node[left, xshift=0.0cm, yshift=0.0cm,fill=orange!25,text width=1.3cm] {{\scriptsize Cell Type (LSTM, GRU)}};

\foreach \m/\l [count=\y] in {1,2,3}
{
	\node [circle,fill=green!50,minimum size=1cm] (input-\m) at (0,2.5-\y) {};
}
\foreach \m/\l [count=\y] in {4}
{
	\node [circle,fill=green!50,minimum size=1cm ] (input-\m) at (0,-2.5) {};
}

\node [neuron missing]  at (0,-1.5) {};


\foreach \m [count=\y] in {1}
\node [circle,fill=red!50,minimum size=1cm ] (hidden-\m) at (2,0.75) {};

\foreach \m [count=\y] in {2}
\node [circle,fill=red!50,minimum size=1cm ] (hidden-\m) at (2,-1.85) {};


\node [neuron missing]  at (2,-0.3) {};


\foreach \m [count=\y] in {1}
\node [circle,fill=blue!50,minimum size=1cm ] (output-\m) at (4,1.5-\y) {};

\foreach \m [count=\y] in {2}
\node [circle,fill=blue!50,minimum size=1cm ] (output-\m) at (4,-0.5-\y) {};

\node [neuron missing]  at (4,-0.4) {};


\foreach \l [count=\i] in {1,2,3,111}
\draw [<-] (input-\i) -- ++(-1,0)
node [above, midway] {$I_{\l}$};

\foreach \l [count=\i] in {1,16}
\node [above] at (hidden-\i.north) {$H_{\l}$};

\foreach \l [count=\i] in {1,16}
\draw [->] (output-\i) -- ++(1,0)
node [above, midway] {$O_{ \l}$};

\foreach \i in {1,...,4}
\foreach \j in {1,...,2}
\draw [->] (input-\i) -- (hidden-\j);

\foreach \i in {1,...,2}
\foreach \j in {1,...,2}
\draw [->] (hidden-\i) -- (output-\j);

\foreach \l [count=\x from 0] in {Input, Hidden (nL), Ouput (nClass)}
\node [align=center, above] at (\x*2,2) {\l \\ layer};

\end{tikzpicture}

		

%		\caption{LSTM internal cell adapted from~\cite{colah}}~\label{Fig:LSTM_Cell_Chaining}
	\end{figure}%
\end{center}

	\end{column}

\begin{column}{.55\textwidth}

			\input{./Figures/Fig_LSTM_Cell_full_not_scaled.tex}
		\input{./Figures/Fig_BI_LSTM.tex}
	\end{column}

\end{columns}
}


\end{frame}

%%%%%%%%%%%%%%%%%%%%%%%%%%%%%%%%%%%%%%%%%%%%%%%%%%

%\begin{frame}[fragile]{LSTM Architectures}
%\begin{center}
%\begin{figure}[t]
%	\minipage{0.8\textwidth}
%	\input{./Figures/Fig_LSTM_Cell_full_not_scaled.tex}
%	\endminipage
%	\caption{LSTM internal cell adapted from~\cite{colah}}~\label{Fig:LSTM_Cell_Chaining}
%\end{figure}%
%\end{center}
%\end{frame}

%%%%%%%%%%%%%%%%%%%%%%%%%%%%%%%%%%%%%%%%%%%%%%%%%%
%%%%%%%%%%%%%%%%%%%%%%%%%%%%%%%%%%%%%%%%%%%%%%%%%%
%%%%%%%%%%%%%%%%%%%%%%%%%%%%%%%%%%%%%%%%%%%%%%%%%%

\section{Experiments and Results}

%%%%%%%%%%%%%%%%%%%%%%%%%%%%%%%%%%%%%%%%%%%%%%%%%%
\begin{frame}[fragile]{Experiments Parameters}
\begin{itemize}
	\item \textbf{Dataset Configurations} (3 $\times$ 2 $\times$ 2):
	\begin{itemize}
		\item [-] Encoding technique (3): BinE, OneE, TwoE.
		\item [-] Diacritics (2): 0D, 1D.
		\item [-] Trimming (2): 0T, 1T.
	\end{itemize}
\item \textbf{Network Configurations} (2 $\times$ 2 $\times$ 2 $\times$ 2):
\begin{itemize}
\item [-] Loss functions (2): \textit{Weighted} or \textit{Non-Weighted } \textbf{(1, 0)} respectively.
\item [-] The number of layers (2): nL.
\item [-] The number of cell units (2): nU.
\item [-] Cell type (2): LSTM, Bi-LSTM.
\end{itemize}
\end{itemize}
\begin{block}{Total Experiements Configurations}
	Dataset Conf. (12) $\times$ Network Conf. (16) = 192 Experiement.
\end{block}
\end{frame}
%%%%%%%%%%%%%%%%%%%%%%%%%%%%%%%%%%%%%%%%%%%%%%%%%%
\begin{frame}[fragile]{Overall Accuracy!}


\begin{figure}[!t]
	\input{./Figures/Fig_Overall_Score.tex}
	\caption{Overall accuracy of the 192 experiments}~\label{Fig:ArabicModelsResults}
\end{figure}

\end{frame}

%%%%%%%%%%%%%%%%%%%%%%%%%%%%%%%%%%%%%%%%%%%%%%%%%%
\begin{frame}[fragile]{Detailed Analysis for Overall Accuracy winner!}
%\Wider{
		\begin{columns}

		\begin{column}{.3\textwidth}
			\begin{table}[h]
				\footnotesize 
				\centering
				\begin{tabular}{c c c}
					\toprule
					\textbf{Ref.}& \textbf{Accuracy}& \textbf{Test Size} \\
					\midrule
					\cite{Alnagdawi2013FindingArabicPoemMeter} & 75\% & 128\\
					\cite{Abuata2016RuleBasedAlgorithm}& 82.2\% & 417\\
					DNN & 96.38\%& 150,000 \\
					\bottomrule
				\end{tabular}
				\caption{{\scriptsize Overall accuracy of this article compared to literature}.}\label{Tab:Summary_Results}
			\end{table}		
		\end{column}
	
			\begin{column}{.7\textwidth}
		\begin{figure}[!t]
			\input{./Figures/Fig_Results_Per_Class.tex}
			\caption{{\scriptsize The per-class accuracy score of the best four models.}}~\label{Fig:ArabicModelsResults}
		\end{figure}
	\end{column}
	
		\end{columns}

%}
\end{frame}


%%%%%%%%%%%%%%%%%%%%%%%%%%%%%%%%%%%%%%%%%%%%%%%%%%


%%%%%%%%%%%%%%%%%%%%%%%%%%%%%%%%%%%%%%%%%%%%%%%%%%
%%%%%%%%%%%%%%%%%%%%%%%%%%%%%%%%%%%%%%%%%%%%%%%%%%
%\begin{frame}[fragile]{Comparison with related works}
%
%\end{frame}


%%%%%%%%%%%%%%%%%%%%%%%%%%%%%%%%%%%%%%%%%%%%%%%%%%
%\begin{frame}[fragile]{Per-class Accuracy!}
%\Wider{
%\begin{figure}[!t]
%	\input{./Figures/Fig_Results_Per_Class.tex}
%	\caption{The per-class accuracy score of the best four models.}~\label{Fig:ArabicModelsResults}
%\end{figure}
%}
%
%\end{frame}
%%%%%%%%%%%%%%%%%%%%%%%%%%%%%%%%%%%%%%%%%%%%%%%%%%

\section{Discussions}


%%%%%%%%%%%%%%%%%%%%%%%%%%%%%%%%%%%%%%%%%%%%%%%%%%
\begin{frame}[fragile]{Encoding effect}
%\Wider{
	\begin{figure}[!t]
		  \begin{tikzpicture}
		\input{./Figures/Fig_Results_Encoding_Convergence.tex}
		  \end{tikzpicture}
		\caption{Encoding effect on Learning rate with the best model (1T, 0D, 4L, 82U, 0W, BinE) and when using the two other encodings	instead of BinE.}~\label{Fig:ArabicModelsResults}

	\end{figure}
%}

\end{frame}

%%%%%%%%%%%%%%%%%%%%%%%%%%%%%%%%%%%%%%%%%%%%%%%%%%


\begin{frame}[fragile]{Encoding effect}

\begin{block}{Encoding}
	\begin{itemize}
		
		\item The encoding method is a transformer function $\mathcal{T}$, which transforms discrete input values $X$. 
		\item It is assumed the network $\eta_1$ is the most accurate network which can ``decode'' $\mathcal{T}(X)$. 
		\item If we have another encoding function $\mathcal{T}_2$ and we try to use the same network $\eta_1$ for the $\mathcal{T}_2$ as $\eta_1\left(\mathcal{T}_1(X)\right) = \left(\eta_1\cdot\mathcal{T}_1\cdot \mathcal{T}_2^{-1} \right)\left(\mathcal{T}_2(X)\right)$. This network may be of complicated architecture to be able to “decode” a terse or complex pattern $\mathcal{T}_2(X)$.
		
	\end{itemize}
	
\end{block}

\end{frame}

%%%%%%%%%%%%%%%%%%%%%%%%%%%%%%%%%%%%%%%%%%%%%%%%%%
\begin{frame}[fragile]{Classifying Arabic Non-Poem Text}
We found 99.2\% of our 150k testing observations which correctly classified have a score range between 0.94 and 1.0.
\begin{center}
	\begin{block}{Arabic Article}
		\begin{Arabic}
	قاد الدولي المصري محمد صلاح فريقه ليفربول للعودة إلى صدارة الدوري الإنجليزي الممتاز، بعد الفوز على ضيفه بورنموث بثلاثية نظيفة، \underline{\textbf{خلال المباراة التي جمتهما مساء السبت بالجولة الـ26 من المسابقة.}}	ونستعرض في التقرير التالي أبرز الأرقام التي حققها صاحب الـ26 عامًا بعد العودة للتسجيل أمام بورنموث: يعد بورنموث بوابة صلاح للعودة للتسجيل هذا الموسم في بريميرليج\\
\end{Arabic}
{\color{blue}{\tiny source: \url{https://www.yallakora.com/epl/2545/News/360950/}}}

	\end{block}

\Wider{
	
\begin{Arabic}
\begin{table}
\small
\begin{tabular}[h!]{c c c c c c c c }
				
	\multicolumn{4}{c} \hfill \textarabic{خــلال المــبـاراة التــي جـمـعـتـهـمـا}     & \multicolumn{4}{c} \hfill \textarabic{مـســاء الســبـت بالجـولـة الـ26 من المـســابـقـة} \hfill
	\\ 
	{\color{black} خلالَلْ} & {\color{black} مُباراتل} & {\color{black} لَتِيجُ} & {\color{black} مَعَتْهُما} & 
	{\color{black} مِساءَسْ} & {\color{black} سَبَتْبِلْجَوْ} & {\color{black} لَتِلْمِنَلْ } & {\color{black} مُسَابَقَه}\\
	{\color{black} \texttt{0/0//}}  &{\color{black} \texttt{0/0/0//}}& {\color{black} \texttt{/0//}} & {\color{black} \texttt{0//0//}} &
	{\color{black} \texttt{0/0//}} & {\color{black} \texttt{0/0/0//}} & \texttt{{\color{black}/0//}{\color{red}/}{\color{black}0}} & {\color{black} \texttt{0//0//}}\\
	
	{\color{black} \texttt{0/0//}} & {\color{black} \texttt{0/0/0// }} & {\color{black} \texttt{/0//}} & {\color{black} \texttt{0/0///}} &
	{\color{black} \texttt{0/0//}} & {\color{black} \texttt{0/0/0//}} & {\color{black} \texttt{0/0//}} & {\color{black} \texttt{0//0//}}\\
	
	
	{\color{black} فَعُوْلُنْ} &  {\color{black} مَفَاعِيْلُنْ} &  {\color{black} فَعُولُ} &  {\color{black} مَفَاعِلُنْ} & 
	{\color{black} فَعُوْلُنْ} &  {\color{black} مَفَاعِيْلُنْ} &  {\color{black} فَعُوْلُنْ} &  {\color{black} مَفَاعِيْلُنْ}\\
%	\hline
\end{tabular}
\end{table}
\end{Arabic}%

}	
\end{center}

\end{frame}

%%%%%%%%%%%%%%%%%%%%%%%%%%%%%%%%%%%%%%%%%%%%%%%%%%


\begin{frame}{Bibliography}

\begin{thebibliography}{10}
	{\footnotesize 
	\bibitem{Abuata2016RuleBasedAlgorithm}
	\alert{(1) Abuata, Belal and Al-Omari, Asma}
	\newblock  {A Rule-Based Algorithm for the Detection of Arud Meter in Classical Arabic Poetry}
	\newblock {\em International Arab Journal of Information Technology. (2017), 15}.
	
	\bibitem{Alnagdawi2013FindingArabicPoemMeter}
	\alert{(2) Alnagdawi, Mohammad and Rashaideh, Hasan and Aburumman, Ala}
	\newblock  {Finding Arabic Poem Meter Using Context Free Grammar}
	\newblock {\em J. of Commun. {\&} Comput. Eng. (2013), 3, 52-59}.
	
	\bibitem{colah}
	\alert{(3) Colah}
	\newblock  {Understanding Lstm Networks}
	\newblock {\em \url{http://colah.github.io/posts/2015-08-Understanding-LSTMs/} , 2015}.
	
	\bibitem{Gitrepo_NN_Tikz}
	\alert{(4) Petar Veličković}
	\newblock  {Collection of Latex Tikz figures}
	\newblock {\em \url{https://github.com/PetarV-/TikZ} }.


	\bibitem{PoetryEncyclopedia2016}
	\alert{(5) \textarabic{المَوسُوعَةُ الشِعْرية}}
	\newblock  {Department of Culture and Tourism – Abu Dhabi}
	\newblock {\em \url{https://poetry.dctabudhabi.ae }}.

	\bibitem{diwan}
	\alert{(6) \textarabic{الدّيْوَانُ}}
	\newblock  {Al-Diwan website}
	\newblock {\em \url{https://www.aldiwan.net} }.
	}
	
\end{thebibliography}
\end{frame}


%%%%%%%%%%%%%%%%%%%%%%%%%%%%%%%%%%%%%%%%%%%%%%%%%%%%%%%%%%%%%%%%%%%%%%%%%%%


%%%%%%%%%%%%%%%%%%%%%%%%%%%%%%%%%%%%%%%%%%%%%%%%%%
\begin{frame}[fragile]{Questions!}
\begin{center}
	Questions.
\end{center}

\end{frame}

\section{Appendex}

%%%%%%%%%%%%%%%%%%%%%%%%%%%%%%%%%%%%%%%%%%%%%%%%%%
\begin{frame}[fragile]{RNN, Architectures}
\begin{figure}[t]
	\minipage{\textwidth}	
	\centering
	\begin{tikzpicture}[scale=1.2]

% The grid
%  \draw[step=0.5, gray!40, very thin] (-1,-1) grid (8,8);

% LSTM frame
\def \yONE {0.75}

\def \sigmoidWidth  {0.5}
\def \sigmoidInDist {0.2}
\def \sigmoidHeight  {0.5}
\def \yB  {1.5}
\def \yD  {2.75}
\def \xZERO  {0}
\def \xONE  {2}
\def \xTWO  {3.5}
\def \xTHREE  {5}
\def \xFOUR  {7}
\def \shift {1}
\def \bigR {0.25cm}
%%%%%%%%%%%%%%%%%%%%%first one 
% First Sigmoid
\draw[fill=LightYellow] (\xONE, \yB) rectangle (\xONE+ \sigmoidWidth, \yB + \sigmoidHeight) node[pos=0.5] {\large$\sigma$};
% Upper Arrow Sigmoid 
\draw[arrows=-latex, line width=1.8pt]  (\xONE + \sigmoidWidth/2, \yB +\sigmoidHeight) -- (\xONE + \sigmoidWidth/2, 2.6);
% Circle X_{t+1}
\draw[thick,fill=LightBlue1] (\xONE + \sigmoidWidth/2, \yONE ) circle (\bigR) node {\large$x_0$};
% Down Arrow X_{t+1}
\draw[arrows=-latex,line width=1.8pt] (\xONE + \sigmoidWidth/2, \yONE + 0.25)  --  (\xONE + \sigmoidWidth/2, \yB);
% Circle H{t} 
\draw[thick,fill=LightPurple1] (\xONE + \sigmoidWidth/2, \yD) circle (\bigR) node {\large$h_0$};
% Right Arrow H{t} 
\draw[arrows=-latex, line width=1.8pt] (\xONE + \sigmoidWidth/2+\sigmoidWidth/2, 1.75)   
--  (\xTWO, 1.75)
node[right, xshift=-0.9cm, yshift=0.4cm,] {\large$h_0$};
%%%%%%%%%%%%%%%%%%%%%second one 
% First Sigmoid
%%%%%%%%%%%%%%%%%%%%%first one 
% First Sigmoid
\draw[fill=LightYellow] (\xTWO, \yB) rectangle (\xTWO+ \sigmoidWidth, \yB + \sigmoidHeight) node[pos=0.5] {\large$\sigma$};
% Upper Arrow Sigmoid 
\draw[arrows=-latex, line width=1.8pt]  (\xTWO + \sigmoidWidth/2, \yB +\sigmoidHeight) -- (\xTWO + \sigmoidWidth/2, 2.6);
% Circle X_{t+1}
\draw[thick,fill=LightBlue1] (\xTWO + \sigmoidWidth/2, \yONE ) circle (\bigR) node {\large$x_1$};
% Down Arrow X_{t+1}
\draw[arrows=-latex,line width=1.8pt] (\xTWO + \sigmoidWidth/2, \yONE + 0.25)  --  (\xTWO + \sigmoidWidth/2, \yB);
% Circle H{t} 
\draw[thick,fill=LightPurple1] (\xTWO + \sigmoidWidth/2, \yD) circle (\bigR) node {\large$h_1$};
% Right Arrow H{t} 
\draw[arrows=-latex, line width=1.8pt] (\xTWO + \sigmoidWidth/2+\sigmoidWidth/2, 1.75)   
--  (\xTHREE, 1.75)
node[right, xshift=-0.9cm, yshift=0.4cm,] {\large$h_1$};

%%%%%%%%%%%%%%%%%%%%%first one 
% First Sigmoid
\draw[fill=LightYellow] (\xTHREE, \yB) rectangle (\xTHREE+ \sigmoidWidth, \yB + \sigmoidHeight) node[pos=0.5] {\large$\sigma$};
% Upper Arrow Sigmoid 
\draw[arrows=-latex, line width=1.8pt]  (\xTHREE + \sigmoidWidth/2, \yB +\sigmoidHeight) -- (\xTHREE + \sigmoidWidth/2, 2.6);
% Circle X_{t+1}
\draw[thick,fill=LightBlue1] (\xTHREE + \sigmoidWidth/2, \yONE ) circle (\bigR) node {\large$x_2$};
% Down Arrow X_{t+1}
\draw[arrows=-latex,line width=1.8pt] (\xTHREE + \sigmoidWidth/2, \yONE + 0.25)  --  (\xTHREE + \sigmoidWidth/2, \yB);
% Circle H{t} 
\draw[thick,fill=LightPurple1] (\xTHREE + \sigmoidWidth/2, \yD) circle (\bigR) node {\large$h_2$};
% Right Arrow H{t} 
\draw[arrows=-latex, line width=1.8pt] (\xTHREE + \sigmoidWidth/2+\sigmoidWidth/2, 1.75)   
--  (\xFOUR, 1.75)
node[right, xshift=-0.9cm, yshift=0.4cm,] {\large$h_2$};

%%%%%%%%%%%%%%%%%%%%%first one 
% First Sigmoid
\draw[fill=LightYellow] (\xFOUR, \yB) rectangle (\xFOUR+ \sigmoidWidth, \yB + \sigmoidHeight) node[pos=0.5] {\large$\sigma$};
% Upper Arrow Sigmoid 
\draw[arrows=-latex, line width=1.8pt]  (\xFOUR + \sigmoidWidth/2, \yB +\sigmoidHeight) -- (\xFOUR + \sigmoidWidth/2, 2.6);
% Circle X_{t+1}
\draw[thick,fill=LightBlue1] (\xFOUR + \sigmoidWidth/2, \yONE ) circle (\bigR) node {\large$x_{t}$};
% Down Arrow X_{t+1}
\draw[arrows=-latex,line width=1.8pt] (\xFOUR + \sigmoidWidth/2, \yONE + 0.25)  --  (\xFOUR + \sigmoidWidth/2, \yB);
% Circle H{t} 
\draw[thick,fill=LightPurple1] (\xFOUR + \sigmoidWidth/2, \yD) circle (\bigR) node {\large$h_t$};
% Right Arrow H{t} 
\draw[arrows=-latex, line width=1.8pt] (\xFOUR + \sigmoidWidth/2+\sigmoidWidth/2, 1.75)   
--  (\xFOUR + \sigmoidWidth + \sigmoidWidth, 1.75)
node[right, xshift=-0.7cm, yshift=0.4cm,] {\large$h_t$};
\node[text width=1cm] at (\xFOUR-.75,1) 
{\large....};

%%%%%%%%%%%%%%%%%%%%%first one 
% First Sigmoid
\draw[fill=LightYellow] (\xZERO, \yB) rectangle (\xZERO+ \sigmoidWidth, \yB + \sigmoidHeight) node[pos=0.5] {\large$\sigma$};
% Upper Arrow Sigmoid 
\draw[arrows=-latex, line width=1.8pt]  (\xZERO + \sigmoidWidth/2, \yB +\sigmoidHeight) -- (\xZERO + \sigmoidWidth/2, 2.6);
% Circle X_{t+1}
\draw[thick,fill=LightBlue1] (\xZERO + \sigmoidWidth/2, \yONE ) circle (\bigR) node {\large$x_0$};
% Down Arrow X_{t+1}
\draw[arrows=-latex,line width=1.8pt] (\xZERO + \sigmoidWidth/2, \yONE + 0.25)  --  (\xZERO + \sigmoidWidth/2, \yB);
% Circle H{t} 
\draw[thick,fill=LightPurple1] (\xZERO + \sigmoidWidth/2, \yD) circle (\bigR) node {\large$h_0$};
% Right Arrow H{t} 
\draw[line width=1.8pt] (\xZERO + \sigmoidWidth/2+\sigmoidWidth/2, 1.75)   
--  (.5+\sigmoidWidth/2, 1.75);

\draw[line width=1.8pt] (.5+\sigmoidWidth/2, 1.73)   
--  (.5+\sigmoidWidth/2, 2.25);

\draw[line width=1.8pt] (.5+\sigmoidWidth/2, 2.23)   
--  (\xZERO + \sigmoidWidth/2+.03, 2.23);

\draw[line width=1.8pt] (\xZERO + \sigmoidWidth/2-.03, 2.23)   
--  (\xZERO - \sigmoidWidth/2, 2.23);

\draw[line width=1.8pt] (\xZERO - \sigmoidWidth/2, 2.25)
--  (\xZERO - \sigmoidWidth/2, 1.75);

\draw[arrows=-latex,line width=1.8pt] (\xZERO-.02 - \sigmoidWidth/2, 1.75)
--  (\xZERO , 1.75);

\node[text width=1cm] at (1.5,1.75) {\large=};
\end{tikzpicture}

	\endminipage\hfill
	\caption{Recurrent Neural Networks Loops adapted from~\cite{colah}}\label{Fig:RNN_Rolled_Loop}
	
\end{figure}%
\end{frame}

%%%%%%%%%%%%%%%%%%%%%%%%%%%%%%%%%%%%%%%%%%%%%%%%%%
\begin{frame}[fragile]{RNN, Architectures}
\begin{figure}[t]
\minipage{0.3\textwidth}
\input{./Figures/Fig_Rnn_Single_Tanh_Left.tex}
\endminipage\hfill
\minipage{0.3\textwidth}
\begin{tikzpicture}


% The grid
% \draw[step=0.5, gray!40, very thin] (0,0) grid (8,4);

% LSTM frame
\def \yONE {0.5}
\def \yTWO {3.3}
\draw[rounded corners=10pt, very thick, fill=LightGreen]  (1.5, 0.5) rectangle (5, 2.3);

\def \sigmoidWidth  {0.45}
\def \sigmoidInDist {0.2}
\def \sigmoidHeight  {0.3}
\def \yB  {1.3}
\def \yD  {2}
\def \shift {0.3}

% Square tanh
\draw[fill=LightYellow]  (3,1.5-\sigmoidHeight/2) 
rectangle 
(3 +\sigmoidWidth+.2, 1.5-\sigmoidHeight/2 +\sigmoidHeight)
node[pos=0.5] {\scriptsize\textit{tanh}};

%tanh upper line
\draw[line width=1.8pt]  (3.2,1.65) -- (3.2, \yD);

% Second times operator and add operator
\def \xTimesB {2.6}
\def \yC      {1}

%line A upper left input from previous cell
%\draw[arrows=-latex,line width=1.8pt]  (1, \yD) -- (1.5, \yD);

%line A upper left input after cell
\draw[line width=1.8pt]  (1.5, \yD) -- (2.385, \yD);

\draw[arrows=-latex, line width=1.8pt] (2.367, \yD)  to[out=270, in=180] (3,1.5);

%line A upper from tanh to rictangle border
\draw[line width=1.8pt]  (3.17, \yD) -- (5, \yD);

%line A upper from rictangle border to next cell
\draw[arrows=-latex,line width=1.8pt]  (4.9, \yD) -- (5.7, \yD);


\def \levelB {1}
%level B line 
\draw[line width=1.8pt] (2.2, \levelB) 
--
(2.385, \levelB);

\def \xOfThirdSigmoid {\xSSS + \sigmoidWidth/2}
%second line last angle connected to sigmoid 

\draw[arrows=-latex, line width=1.8pt] (2.367,1) 
to[out=90, in=180] 
(3,1.5);


\def \bigR {0.3cm}
\def \bigRTWO {0.4cm}
\def \C    {1}
\def \xBigCicleONE {2}
\def \xBigCicleTWO {5.5 + 0.25}

\draw[thick,fill=LightBlue] (\xBigCicleONE, \yONE+.4 - \C) circle (\bigR) node {$x_{t}$};

% x_t line
\draw[line width=1.8pt] (\xBigCicleONE, \yONE -\C +0.7) 
--
(\xBigCicleONE, \yONE -\C +0.25 + 1.1);

\draw[line width=1.8pt] (\xBigCicleONE, \yONE -\C +0.25 + 1.1)
to[out=90, in=180]
(2.25, 1);



\draw[thick,fill=LightPurple] (\xBigCicleTWO-1, \yTWO -1.15 + \C) circle (\bigR) node {$h_{t}$};
%h_T lines
\draw[arrows=-latex, line width=1.8pt]
(\xBigCicleTWO-1, \yD )
--
(\xBigCicleTWO-1, \yTWO -1.4 + \C);


\end{tikzpicture}

\endminipage\hfill
\minipage{0.3\textwidth}%
\begin{tikzpicture}[scale=1]


% The grid
% \draw[step=0.5, gray!40, very thin] (0,0) grid (8,4);

% LSTM frame
\def \yONE {0.5}
\def \yTWO {3.3}
\draw[rounded corners=10pt, very thick, fill=LightGreen]  (1.5, 0.5) rectangle (5, 2.3);

\def \sigmoidWidth  {0.45}
\def \sigmoidInDist {0.2}
\def \sigmoidHeight  {0.3}
\def \yB  {1.3}
\def \yD  {2}
\def \shift {0.3}

% Square tanh
\draw[fill=LightYellow]  (3,1.5-\sigmoidHeight/2) 
rectangle 
(3 +\sigmoidWidth+.2, 1.5-\sigmoidHeight/2 +\sigmoidHeight)
node[pos=0.5] {\scriptsize\textit{tanh}};

%tanh upper line
\draw[line width=1.8pt,gray]  (3.2,1.65) -- (3.2, \yD);

% Second times operator and add operator
\def \xTimesB {2.6}
\def \yC      {1}

%line A upper left input from previous cell
%\draw[arrows=-latex,line width=1.8pt,gray]  (1, \yD) -- (1.5, \yD);

%line A upper left input after cell
\draw[line width=1.8pt,gray]  (1.5, \yD) -- (2.385, \yD);

\draw[arrows=-latex, line width=1.8pt,gray] (2.367, \yD)  to[out=270, in=180] (3,1.5);

%line A upper from tanh to rictangle border
\draw[line width=1.8pt,gray]  (3.17, \yD) -- (5, \yD);

%line A upper from rictangle border to next cell
\draw[arrows=-latex,line width=1.8pt]  (5, \yD) -- (5.5, \yD);


\def \levelB {1}
%level B line 
\draw[line width=1.8pt,gray] (2.2, \levelB) 
--
(2.385, \levelB);

\def \xOfThirdSigmoid {\xSSS + \sigmoidWidth/2}
%second line last angle connected to sigmoid 

\draw[arrows=-latex, line width=1.8pt,gray] (2.367,1) 
to[out=90, in=180] 
(3,1.5);


\def \bigR {0.3cm}
\def \bigRTWO {0.4cm}
\def \C    {1}
\def \xBigCicleONE {2}
\def \xBigCicleTWO {5.5 + 0.25}

\draw[thick,fill=LightBlue] (\xBigCicleONE, \yONE+.4 - \C) circle (\bigR) node {{\scriptsize$x_{t+1}$}};

% x_t line
\draw[line width=1.8pt] (\xBigCicleONE, \yONE -\C +0.7) 
--
(\xBigCicleONE, \yONE );

\draw[line width=1.8pt,gray] (\xBigCicleONE, \yONE ) -- (\xBigCicleONE, \yONE  + .37);

\draw[line width=1.8pt,gray] (\xBigCicleONE, \yONE -\C +0.25 + 1.1)
to[out=90, in=180]
(2.25, 1);



\draw[thick,fill=LightPurple] (\xBigCicleTWO-1, \yTWO -1.15 + \C) circle (\bigR) node {{\scriptsize$h_{t+1}$}};
%h_T lines
\draw[arrows=-latex, line width=1.8pt]
(\xBigCicleTWO-1, \yD+.3 )
--
(\xBigCicleTWO-1, \yTWO -1.4 + \C);


\draw[line width=1.8pt,gray]
(\xBigCicleTWO-1, \yD)
--
(\xBigCicleTWO-1, \yD+.3 );


\end{tikzpicture}

\endminipage
\caption{A single recurrent layer adapted from~\cite{colah}}~\label{Fig:LSTM_SimpleRNN}
\end{figure}%
\end{frame}

%%%%%%%%%%%%%%%%%%%%%%%%%%%%%%%%%%%%%%%%%%%%%%%%%%

\begin{frame}[fragile]{LSTM Architectures}
\begin{figure}[t]
\minipage{0.32\textwidth}
\input{./Figures/Fig_LSTM_Cell_Left.tex}
\endminipage\hfill
\minipage{0.32\textwidth}
\input{./Figures/Fig_LSTM_Cell_full.tex}
\endminipage\hfill
\minipage{0.33\textwidth}%
\input{./Figures/Fig_LSTM_Cell_Right.tex}
\endminipage
\caption{Unfold LSTM adapted from~\cite{colah}}
~\label{Fig:LSTM_Cell_Chaining}
\end{figure}%
\end{frame}

%%%%%%%%%%%%%%%%%%%%%%%%%%%%%%%%%%%%%%%%%%%%%%%%%%

\begin{frame}[fragile]{LSTM Architectures}
\begin{block}{Bi-LSTM Motivation}
\begin{itemize}
\item[--] \textit{\textbf{Harry}} is the king, and he will travel next week.
\item[--] The new book which makes the big sale is named \textit{\textbf{Harry}} Potter.
\end{itemize}
\end{block}

\begin{itemize}
\item Bi-LSTM models always outperform LSTM models.
\item This means that models cannot learn the pattern from one direction; it
should be two directions together.
\end{itemize}
\end{frame}
%%%%%%%%%%%%%%%%%%%%%%%%%%%%%%%%%%%%%%%%%%%%%%%%%%


		\Large 
		\begin{tabular}[h!]{c c c c c c c c }
			
			\multicolumn{4}{c} \hfill \textarabic{تدفق في البطحاء بعد تبهطل}     & \multicolumn{4}{c} \hfill \textarabic{وقعقع في البيداءغير مزركلِ} \hfill
			\\ 

			\multicolumn{4}{c} \hfill \textarabic{وسار بأركان العقيش مقرنصا}     & \multicolumn{4}{c} \hfill \textarabic{وهـــام بكـل القارطات بشنكلِ} \hfill
			\\ 
			\multicolumn{4}{c} \hfill \textarabic{يقول: وما بال البحاط مقرمطا}     & \multicolumn{4}{c} \hfill \textarabic{ويسعى دواما بين هـك وهنكلِ} \hfill
			\\ 
			\multicolumn{4}{c} \hfill \textarabic{إذا أقبـل البحراط طـاح بهمـة}     & \multicolumn{4}{c} \hfill \textarabic{وإن أقرط البحطوش ناء بكلكلِ} \hfill
			\\ 
			\multicolumn{4}{c} \hfill \textarabic{يكاد على فرط الحطيف يبقبق}     & \multicolumn{4}{c} \hfill \textarabic{ويضرب ما بين الهماط وكندلِ} \hfill
			\\ 
		\end{tabular}
	
	\end{table}
\end{Arabic}%
\end{frame}

%%%%%%%%%%%%%%%%%%%%%%%%%%%%%%%%%%%%%%%%%%%%%%%%%%

%%% Local Variables:
%%% mode: latex
%%% TeX-master: "../main"
%%% TeX-engine: xetex
%%% End:
